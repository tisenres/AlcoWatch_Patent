%%%%%%%%%%%%%%%%%%%%%%%%%%%%%%%%%%%%%%%%%%%%%%%%%%%%%%%%%%%%%%%%%%%%%%%%%%%%
%% Trim Size: 9.75in x 6.5in
%% Text Area: 8in (include Runningheads) x 5in
%% ws-jcsc.tex : 01-04-2024
%% Tex file to use with ws-jcsc.cls written in Latex2E.
%% The content, structure, format and layout of this style file is the
%% property of World Scientific Publishing Co. Pte. Ltd.
%% Copyright 2024 by World Scientific Publishing Co.
%% All rights are reserved.
%%%%%%%%%%%%%%%%%%%%%%%%%%%%%%%%%%%%%%%%%%%%%%%%%%%%%%%%%%%%%%%%%%%%%%%%%%%%
%%

%\documentclass[wsdraft]{ws-jcsc}
\documentclass{ws-jcsc}
\usepackage{xcolor}
\usepackage[verbose]{hyperref}
\hypersetup{colorlinks=false,allbordercolors=blue,pdfborderstyle={/S/U/W 1}}
% \label, \ref and \cite commands are highly recommended

\begin{document}

\markboth{Authors' Names}{Instructions for Typesetting Manuscripts (Condensed Title for the Paper)}

%%%%%%%%%%%%%%%%%%%%% Publisher's Area please ignore %%%%%%%%%%%%%%%
%
\catchline{}{}{}{}{}
%
%%%%%%%%%%%%%%%%%%%%%%%%%%%%%%%%%%%%%%%%%%%%%%%%%%%%%%%%%%%%%%%%%%%%

\title{Instructions for Typesetting Manuscripts\\
Using \LaTeX\footnote{For the
title, try not to use more than 3 lines.
Typeset the title in 10~pt Times roman, boldface with the first letter of important words capitalized.}}

\author{First Author\footnote{Typeset names in
8~pt Roman. Use the footnote to indicate the
present or permanent address of the author.}}

\address{University Department, University Name, Address,\\
City, State ZIP/Zone,
Country\footnote{State completely without abbreviations, the
affiliation and mailing address, including country. Typeset in
8~pt Times Italic.}\\
$^\dag$firstauthor\_id@domain\_name}

\author{Second Author$^\S$}

\address{Group, Laboratory, Address,\\
City, State ZIP/Zone, Country\\
$^\S$secondauthor\_id@domain\_name}

\maketitle

\begin{history}
\received{(Day Month Year)}
\revised{(Day Month Year)}
\accepted{(Day Month Year)}
\published{(Day Month Year)}
%\comby{(xxxxxxxxxx)}
\end{history}

\begin{abstract}
The abstract should summarize the context, content and conclusions of
the paper in less than 200 words. It should not contain any references
or displayed equations.
Typeset the abstract in 8 pt Times Roman with baselineskip of 10 pt,
making an indentation of 0.25 inches on the left and right margins.
Typeset similarly for keywords below.
\end{abstract}

\keywords{Keyword1; keyword2; keyword3.}

%\ccode{2020 Mathematics Subject Classification: 22E46, 53C35, 57S20}

\section{General Appearance}
Contributions to {\it Journal of Circuits, Systems, and Computers} are to be in American English.
Authors are encouraged to have their contribution checked for grammar.
American spelling should be used.
Abbreviations are allowed but should be spelt out in full when first used.
Integers ten and below are to be spelt out.
Italicize foreign language phrases (e.g.,~Latin, French).
Upon acceptance, authors are required to submit their data source file including postscript files for figures.

The text is to be typeset in 10 pt Times Roman, single spaced with baselineskip of 13~pt.
Text area (including copyright block) is 8 inches high and 5 inches wide for the first page.
Text area (excluding running title) is 7.7 inches high and 5 inches wide for subsequent pages.
Final pagination and insertion of running titles will be done by the publisher.

\section{Major Headings}
Major headings should be typeset in boldface with the first letter of important words capitalized.

\subsection{Sub-headings}
Sub-headings should be typeset in boldface italic and capitalize
the first letter of the first word only. Section number to be in
boldface roman.

\subsubsection{Sub-subheadings}
Typeset sub-subheadings in medium face italic and capitalize the
first letter of the first word only. Section number to be in roman.

\subsection{Numbering and spacing}
Sections, sub-sections and sub-subsections are numbered in Arabic.
Use double spacing before all section headings, and single spacing after section headings.
Flush left all paragraphs that follow after section headings.

\subsection{Lists of items}
Items may be numbered in lowercase Roman numerals: One can define
the following list formats to be new environments.

\begin{romanlist}[(ii)]
\item item one
\item item two
	\begin{alphlist}[(b)]
	\item lists within lists can be numbered with lowercase
              roman letters,
	\item second item.
	\end{alphlist}
\end{romanlist}

\section{Equations}
Displayed equations should be numbered consecutively in the paper,
with the number set flush right and enclosed in parentheses.
\begin{equation}
\mu(n, t) = \frac{\sum^\infty_{i=1} 1(d_i < t, N(d_i) = n)}{
\int^t_{\sigma=0} 1(N(\sigma) = n)d\sigma}\,. \label{eq1}
\end{equation}

Equations should be referred to in abbreviated form, e.g.,~``Eq.~(\ref{eq1})''.
In multiple-line equations, the number should be given on the last line.

Displayed equations are to be centered on the page width.
Standard English letters like x are to appear as $x$ (italicized)
in the text if they are used as mathematical symbols.
Punctuation marks are used at the end of equations as if they appeared
directly in the text.

For multiline equations, prefer using \verb|align| or \verb|align*| instead of \verb|eqnarray|.
\begin{align}
    g(x) & = \sum_{k=1}^{N} y_k g(x_k) \nonumber\\
& = \sum_{k=1}^{N} y_k \sum_{l=1}^{M} b_{kl} w_l \nonumber\\
& = \sum_{k=1}^{N} \sum_{l=1}^{M} b_{kl} y_k w_l
\end{align}

\begin{theorem}
Theorems, lemmas, etc. are to be numbered consecutively in the paper.
Use double spacing before and after theorems, lemmas, etc.
\end{theorem}

\begin{proof}
Proofs should end with square box.
\end{proof}

\section{Illustrations and Photographs}
Figures are to be inserted in the text nearest their first reference.
If the author requires the publisher to reduce the figures, ensure that
the figures (including letterings and numbers) are large enough
to be clearly seen after reduction. If\break photographs are to be
used, only black and white ones are acceptable.

\begin{figure}[th]
\centerline{\includegraphics[width=2.0in]{jcscf1}}
\vspace*{6pt}
\caption{A schematic illustration of dissociative recombination. The
direct mechanism, 4m$^2_\pi$ is initiated when the
molecular ion $S_{L}$ captures an electron with kinetic energy.}
\end{figure}

Figures are to be sequentially numbered in Arabic numerals.
The caption must be placed below the figure.
Typeset in 8 pt Times Roman with baselineskip of 10~pt.
Use double spacing between a caption and the text that follows immediately.

Previously published material must be accompanied by written permission from the author and publisher.

\section{Tables}
Tables should be inserted in the text as close to the point of reference as possible.
Some space should be left above and below the table.

\begin{table}[ht]
\tbl{Comparison of acoustic for frequencies for piston-cylinder problem.}
{\begin{tabular}{@{}cccc@{}} \toprule
Piston mass & Analytical frequency & TRIA6-$S_1$ model &
\% Error \\
& (Rad/s) & (Rad/s) \\ \colrule
1.0\hphantom{00} & \hphantom{0}281.0 & \hphantom{0}280.81 & 0.07 \\
0.1\hphantom{00} & \hphantom{0}876.0 & \hphantom{0}875.74 & 0.03 \\
0.01\hphantom{0} & 2441.0 & 2441.0\hphantom{0} & 0.0\hphantom{0} \\
0.001 & 4130.0 & 4129.3\hphantom{0} & 0.16\\ \botrule
\end{tabular}}
\begin{tabnote}
Table notes
\end{tabnote}
\begin{tabfootnote}
\tabmark{a} Table footnote A\\
\tabmark{b} Table footnote B
\end{tabfootnote}
\end{table}

\noindent
Tables should be numbered sequentially in the text in Arabic numerals.
Captions are to be centralized above the tables.
Typeset tables and captions in 8 pt Times Roman with baselineskip of 10~pt.

If tables need to extend over to a second page, the continuation
of the table should be preceded by a caption,
e.g.,~``Table~1 ({\it Continued\/}).''

\section{References}
References in the text are to be numbered consecutively in Arabic
numerals, in the order of first appearance.  They are to be typed
in superscripts after punctuation marks, e.g.,~``$\ldots$ in the
statement''.\cite{1}

\section{Footnotes}
Footnotes should be numbered sequentially in superscript
lowercase Roman letters.\footnote{Footnotes should be
typeset in 8 pt Times Roman at the bottom of the page.}

\section*{Acknowledgments}
This section should come before the References.
Funding information may also be included here.

\section*{ORCID}
You are encouraged to include in your user information the ORCID (\url{https://orcid.org/})
or register for one if you don't have it.
This ID will help to identify you in the researcher community and make it easier to keep track of all your publications.
Please provide a valid ORCID here, e.g.,

\noindent Josiah Carberry - \url{https://orcid.org/0000-0002-1825-0097}

\noindent Rajesh Babu - \url{https://orcid.org/0009-0006-0415-6880}

\appendix

\section{Appendices}
Appendices should be used only when absolutely necessary.
They should come before the References. If there is more than one
appendix, number them alphabetically. Number displayed equations
occurring in the Appendix in this way, e.g.,~(\ref{appeqn}), (A.2), etc.
\begin{equation}
\mu(n, t) = {\sum^\infty_{i=1} 1(d_i < t, N(d_i) = n) \over
\int^t_{\sigma=0} 1(N(\sigma) = n)d\sigma}\,.
\label{appeqn}
\end{equation}

\section*{References}
References are to be listed in the order cited in the text. Use
the style shown in the following examples. For journal names, use
the standard abbreviations. Typeset references in 9 pt Times
Roman.

\begin{thebibliography}{0}

\bibitem{1} R. Lorentz and D. B. Benson, Deterministic and
nondeterministic flow-chart interpretations, {\it J. Comput.
Syst. Sci.} {\bf 27} (1983) 400--433, \url{https://doi.org/10.1016/0022-0000(83)90050-8}.

\bibitem{2} M. J. Beeson, {\it Foundations of Constructive Mathematics}
(Springer, Berlin, 1985), p.~210.

\bibitem{3} K. L. Clark, Negations as failure, {\it Logic and Data
Bases}, eds. H. Gallaire and J. Winker (Plenum Press, New York,
1973), pp.~293--306.

\bibitem{4} M. Joliat, A simple technique for partial elimination
of unit productions from LR($k$) parsers, {\it IEEE Trans.
Comput.} {\bf 27} (1976) 753--764, \url{https://doi.org/10.1109/TC.1976.1674686}.

\bibitem{5} D. Dolve, Unanimity in an unknown and
unreliable environment, {\it Proc. 22nd Annual Symp.
Foundations of Computer Science}, Nashville, TN, October 1981,
pp.~159--168, \url{https://doi.org/10.1109/SFCS.1981.53}.

\bibitem{6} R. Tamassia, C. Batini and M. Talamo, An algorithm for
automatic layout of entity relationship diagrams,
{\it Entity-Relationship Approach to Software Engineering,
Proc. 3rd Int. Conf. Entity-Relationship Approach},
eds. C. G. Davis, S. Jajodia, P. A. Ng and R. T. Yeh
(North-Holland, Amsterdam, 1983), pp.~421--439.

\bibitem{7} W. L. Gewirtz, Investigations in the theory of
descriptive complexity, Ph.D. thesis, New York University (1974).
\end{thebibliography}

\end{document} 